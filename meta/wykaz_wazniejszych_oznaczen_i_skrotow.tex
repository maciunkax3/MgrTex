% http://tex.stackexchange.com/questions/30122/generate-table-of-contents-when-section-sections-without-numbering-has-been

\clearpage
\section*{Wykaz ważniejszych oznaczeń i skrótów} % section* - ukrywa numerowanie oraz wyklucza ze spisu tresci
\addcontentsline{toc}{section}{Wykaz ważniejszych oznaczeń i skrótów}% % reczne dodanie do spisu tresci

\begin{description}[leftmargin=2.5cm,labelwidth=2cm]
\item[API] -- z ang. Application Programming Interface - Zdefiniowany interfejs, określający sposób komunikowania między programami.
\item[CPU] -- z ang. Central Processing Unit - Główny procesor urządzenia
\item[GPU] -- z ang. Central Processing Unit - Procesor graficzny
\item[GWS] -- z ang. Global Work Size - Rozmiar globalnej grupy, w ramach której wykonywana będzie funkcja na procesorze graficznym.
\item[LWS] -- z ang. Local Work Size - Rozmiar lokalnej grupy, w ramach której wykonywana bedzie funkcja na procesorze graficznym.
\item[NV21] -- Format obrazu otrzymanego z podglądu kamery, jeden z formatów typu YUV. 
\item[RGBA] -- Model formatowania obrazka, w którym kanały opisujące kolory czerwony, zielony i niebieski, są opisane oddzielnymi wartościami.
\item[SIMD] -- z ang. Single Instruction Multiple Data - Model wykonywania operacji, w którym pojedyncza instrukcja procesora wykonuje zadanie na wektorze danych.
\item[SPIRV] -- Binarny język pośredni, zdefiniowany przez grupę Khronos. Funkcja w takim kodzie może być wykonana przez api OpenGL, OpenCL czy Vulkan.
\item[Work Grupa] -- Grupa pojedynczych elementów wykonujących funkcje na GPU, w ramach takiej grupy funkcje mogą się synchronizować i współdzielić pamięć. 
\item[Work Item] -- Pojedynczy element wykonujący funkcję na GPU.
\item[YUV] -- Model formatowania obrazu w którym Y to jasność a UV koduję barwę.
\end{description}


