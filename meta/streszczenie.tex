\section*{Streszczenie}
Dzisiejsze inteligentne telefony posiadają coraz więcej cech, które dotychczas przypisywane były wyłącznie komputerom osobistym. Urządzenia te budowane są z coraz lepszych i bardziej wydajnych komponentów. Wszystko wskazuje na to, że nowoczesne aparaty komórkowe zastąpią zwykłe komputery. Do zadań takich jak obróbka obrazu, rozpoznawania mowy czy detekcji obiektów, potrzebna jest duża moc obliczeniowa. W tej pracy przeanalizowano jaką wydajność na urządzeniach mobilnych osiąga środowisko OpenCL, które jest biblioteką dedykowaną do szybkich równoległych obliczeń. W niniejszym dokumencie sprawdzono w jaki sposób inteligentne telefony mogą dostępować biblioteki, która jak się okazało nie jest oficjalnie wspierana na platformie Android oraz pokrótce opisano badaną biblioteki. W ramach pracy przetestowano to api, w kontekście osiąganej mocy obliczeniowej, prędkości przepływu pamięci, szybkości rozmnażania macierzy oraz możliwości obróbki obrazu z kamery w czasie rzeczywistym. Uzyskane wyniki zostały przedstawione w formacie wykresów, opisane i przeanalizowane. Omówione zostały aplikacje, które wykorzystują to api między innymi do uczenia głębokich sieci neuronowych czy obróbki obrazu. 
