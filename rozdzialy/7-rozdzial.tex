\section[Podsumowanie]{Podsumowanie}
Celem niniejszej pracy była analiza środowiska jakim jest OpenCL na platformach mobilnych. Sprawdzono, że z pośród najpopularniejszych systemów mobilnych, tylko Android posiada możliwość obsługi tego api. Przy realizacji zadania sprawdzono, w jaki sposób programy, mogą łączyć się z biblioteką implementującą interfejs i jak to wpływa na ich przenośność. W ramach pracy skonstruowano i wykonano kilka testów mających za zadanie przetestować możliwości urządzeń z systemem Android w pracy z biblioteką OpenCL, pod kątem wydajnościowym, jak i współpracy api OpenCL z OpenGL przy wyświetlaniu obrazu z kamery urządzenia. Testy zostały przeprowadzone na urządzeniach opisanych i porównanych w niniejszym dokumencie. Wykonane aplikacje zostały dokładnie opisane, a ich wyniki w formie wykresów zostały przedstawione w pracy i przeanalizowane. W ramach pracy został wykonane poszukiwanie aplikacji wykorzystujących api OpenCL.

Przy realizacji pracy, problemem okazał się brak oficjalnego wsparcia dla OpenCL przez system Android. Pomimo to producenci procesorów graficznych na urządzenia mobilne, tworzą oprogramowanie, które implementuje funkcjonalość api OpenCL. By móc skorzystać z biblioteki należy albo bezpośrednio połączyć aplikację z konkretną dynamiczną biblioteką, albo ładować ją w trakcie wykonywania kodu. Przy zastosowaniu pierwszego rozwiązania aplikacja może być wykonywana tylko na urządzeniu z którego pochodzi biblioteka, na stałe połączona z programem. W drugim przypadku biblioteka jest ładowana dynamicznie. Interfejs OpenCL na platformie Android dostępny jest jedynie z poziomu natywnego kodu w C++, by używać jej w kodzie aplikacji, niezbędne jest wykonywanie części OpenCL po stronie natywnej biblioteki, lub wykonanie dodatkowej biblioteki, która będzie nakładką na funkcje w C++. 

W ramach przeprowadzonych testów sprawdzono jaką moc obliczeniową mają poszczególne urządzenia przy użyciu api OpenCL. Wyraźnie zaobserwowano, że urządzenia wyprodukowane przez Adreno, reagują podobnie na zmianę typu danych i w raz z używaniem większych typów wektorowych ich moc obliczeniowa rosła, natomiast w przypadku procesorów Mali przy zastosowaniu typów wektorowych moc obliczeniowa pozostawała na tym samym poziomie. Procesory Mali są w stanie wykonywać więcej obliczeń na typach całkowitych, natomiast układy Adreno lepiej radza sobie z operacjami na typach zmiennoprzecinkowych. Zweryfikowano, że rozmiary lokalnych work grup znacząco wpływają na czas wykonywania mnożenia macierzy, które wymaga czytania dalekich od siebie komórek pamięci. Urządzenia najlepsze czasy osiągały gdy rozmiary grup w wymiarze X i Y są równomierne. W przypadku takich grup podczas wykonywania wątki w ramach simd, mogły dostępować elementy z pamięci podręcznej. W takim przypadku kosztowne transfery alokacji do pamięci cache występują rzadziej. W tym teście kluczową rolą jest szybki dostęp do pamięci, dlatego im urządzenia miały lepszy typ pamięci, tym lepiej radziły sobie z tym zadaniem. 

W ramach pracy, sprawdzono możliwości urządzeń do modyfikacji podglądu z kamery przy pomocy OpenCL. Okazało się, że podgląd z kamery dostarczany jest w formacie NV21, obsługi którego api OpenCL nie definiuje. Dlatego by móc użyć obrazu z kamery w wykonywanym kernelu, należało przetransformować go do formatu RGBA. Konwersja znacząco wpłynęła jedynie na urządzenie z procesorem graficznym Mali. Na urządzeniach z GPU od Adreno proces zamiany obrazka na format RGBA i wpisanie go do wyświetlanej textury nie wpłynął na ilość wyświetlanych klatek na sekundę. Po nałożeniu prostych filtrów jak zamianę na obraz w skali szarości lub filtr max rgb, liczba prezentowanych obrazów na sekundę nie zmieniła się. Natomiast użycie filtru uśredniającego, w którym do wyliczenia wartości jednego piksela potrzebne było odczytanie i uśrednienie wartości 25 pikseli, znacząco obniża ilość wyświetlanych klatek, dla niektórych urządzeń nawet o ponad połowę.

Istnieje mało informacji na temat korzystania z api OpenCL w aplikacjach użytkowych. Każdy z producentów procesorów graficznych dla urządzeń z systemem Android, posiada implementację takiego sterownika, co pozwala przypuszczać, że jest on częściej używany niż tylko do testów sprawdzających możliwości urządzenia. Problem z dotarciem do informacji na temat używania tego api przez aplikacje najprawdopodobniej wynika z tego, że twórcy programów zazwyczaj nie umieszczają nigdzie listy wszystkich użytych api, frameworków, czy wersji języka programowania.

W przypadku kontynuowaniu pracy nad tematem, można by było znaleźć sposób na sprawdzenie zużycia energii przy wykonywaniu kerneli. Innym kierunkiem kontynuowania prac mogłoby być sprawdzenie jak OpenCL na platformie Android działa na urządzeniach z kartami graficznymi Intela Amd czy Nvidi.
