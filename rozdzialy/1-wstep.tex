\section{Wprowadzenie}
W dzisiejszym świecie nowoczesne telefony przejmują coraz więcej zadań komputerów osobistych. Urządzenia mobilne stają się coraz bardziej wszechstronne, a ich specyfikacja techniczna coraz mniej odstaje od układów stosowanych w zwykłych komputerach. Aktualne trendy takie jak telefony ze składanym ekranem, widoki pulpitów komputerowych, czy stacje dokujące do telefonów, nastrajają na coraz większe zastępowanie komputerów osobistych na rzecz inteligentnych telefonów. Jedną z przewag jaką aktualnie mają komputery jest możliwość zrównoleglania obliczeń na dyskretnych kartach graficznych. 

\subsection{Cel pracy}
Jednym z api, które ma możliwość wykonywania obliczeń na procesorach graficznych jest OpenCL. Celem tej pracy jest przeanalizowanie działania tej biblioteki na urządzeniach mobilnych. OpenCL jest api nie dostępną na systemie macOS, natomiast na Androidzie nie posiada oficjalnego wsparcia. Pomimo tego producenci procesorów graficznych używanych w telefonach z systemem Android, dostarczają sterownik z implementacją tej biblioteki. W niniejszej pracy urządzenia z systemem Android przetestowane zostaną pod względem mocy obliczeniowej, czasu wykonywania operacji matematycznych i przetwarzaniu obrazu.

\subsection{Zakres pracy}
W ramach niniejszej pracy w drugim rozdziale opisano potrzebne kroki do połączenia aplikacji na Android z biblioteką OpenCL, dalej pokazano przykładowy przebieg prostej aplikacji oraz co potrzebne jest do współdzielenia zasobów z api OpenGL. 

W rozdziale trzecim rozdziale porównane zostały główne wady i zalety procesorów graficznych dwóch najbardziej popularnych producentów, a następnie opisane zostały urządzenia, które w ramach analizy wykonanej na potrzeby tej pracy, zostały wykorzystane do testowania.

W rozdziale numer cztery opisane były testy, które wykorzystują OpenCL na Androidzie. Pokazane zostały kernele wykorzystane w testach oraz opisano szczegóły dotyczące specyfiki przeprowadzanych testów. Przedstawione w rozdziale testy dotyczą między innymi osiągów mocy obliczeniowej, szybkości transferów pamięci, czasu wykonywania operacji matematycznej takiej jak mnożenie macierzy oraz używanie OpenCL do nakładania filtrów na wyświetlanym obrazie z kamery.

W rozdziale piątym przedstawiono wyniki jakie uzyskały urządzenia opisane w rozdziale trzecim podczas wykonywania testów opisanych w rozdziale czwartym. Uzyskane wyniki przedstawiono za pomocą wykresów a osiągnięte rezultaty zostały opisane i przeanalizowane.

W rozdziale szóstym, przedstawione zostały programy, które na systemie Android kożystają z biblioteki OpenCL.
