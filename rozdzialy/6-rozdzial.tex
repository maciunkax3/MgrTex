\section[Aplikacje wykorzystujące OpenCl Przypadki Użycia]{Aplikacje wykorzystujące OpenCl Przypadki Użycia}
OpenCL jest biblioteką, która z pewnością może znaleźć zastosowanie w wielu aplikacjach z systemem Android. Brak oficjalnego wsparcia api na Androidzie sprawia, że ilość informacji o używaniu tej biblioteki przez programy na platforme Android jest znikoma. O używaniu tego api przez aplikacje można znaleźć tylkokilka wzmianek w internecie. W poniższych podrozdzialach opisano wykożystanie tej biblioteki w kilku aplikacjach.
\subsection[Tensorflow]{Tensorflow}
Tensorflow to biblioteka, wykorzystywana do uczenia maszynowego. Tensorflow jest często wykorzystywany do treningów głębokich sieci neuronowych. Proces uczenia sieci neuronowych, wymaga wielokrotnego przemnażania wielu macierzy. Samo wykorzystanie wyuczonego modelu również wiąże się z procesem mnożenia macierzy. Tensorflow to oprogramowanie potrafiące wykożystać różne api służące do obliczeni, takie jak Cuda, Sycl czy OpenCL. W przypadku wersji programu na system Android głównym api używanym przez tensorflow jest OpenCL. Jak wynika z artykułu \cite{Tensor}, użycie api OpenCL umożliwia poprawienie wydajności o 100\% w stosunku do uzywania OpenGL. Przy optymalizacji programu używająć OpenCL pomocne były narzędzia dostarczone przez producenta GPU Adreno. Poniewaz api OpenCL nie jest oficjalnie wspierane, aplikacja tensorflow na starcie odpytuje urządzenie o dostępność biblioteki. W przypadku gdy biblioteka ta jest nie dostępna ładowany jest silnik działający w oparciu o api OpenGL.
\subsection[Adobe]{Adobe}
\subsection[Testy mierzące wydajność i jakość urządzenia]{Testy mierzące wydajność i jakość urządzenia}
