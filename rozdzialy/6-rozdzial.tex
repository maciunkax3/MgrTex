\section[Aplikacje wykorzystujące OpenCl Przypadki Użycia]{Aplikacje wykorzystujące OpenCl Przypadki Użycia}
OpenCL jest biblioteką, która z pewnością może znaleźć zastosowanie w wielu aplikacjach z systemem Android. Brak oficjalnego wsparcia api na Androidzie sprawia, że ilość informacji o używaniu tej biblioteki przez programy na platforme Android jest znikoma. O używaniu tego api przez aplikacje można znaleźć tylkokilka wzmianek w internecie. W poniższych podrozdzialach opisano wykożystanie tej biblioteki w kilku aplikacjach.
\subsection[Tensorflow]{Tensorflow}
Tensorflow to biblioteka, wykorzystywana do uczenia maszynowego. Tensorflow jest często wykorzystywany do treningów głębokich sieci neuronowych. Proces uczenia sieci neuronowych, wymaga wielokrotnego przemnażania wielu macierzy. Samo wykorzystanie wyuczonego modelu również wiąże się z procesem mnożenia macierzy. Tensorflow to oprogramowanie potrafiące wykożystać różne api służące do obliczeni, takie jak Cuda, Sycl czy OpenCL. W przypadku wersji programu na system Android głównym api używanym przez tensorflow jest OpenCL. Jak wynika z artykułu \emph{Even Faster Mobile GPU Inference with OpenCL} \cite{Tensor}, użycie api OpenCL umożliwia poprawienie wydajności o 100\% w stosunku do uzywania OpenGL. Przy optymalizacji programu używająć OpenCL pomocne były narzędzia dostarczone przez producenta GPU Adreno. Poniewaz api OpenCL nie jest oficjalnie wspierane, aplikacja tensorflow na starcie odpytuje urządzenie o dostępność biblioteki. W przypadku gdy biblioteka ta jest nie dostępna ładowany jest silnik działający w oparciu o api OpenGL.
\subsection[OpenCV]{OpenCV}
OpenCV (z ang. Open Source Computer Vision Library), to bibilioteka, która ta implementuje wiele funkcjonalności służących mp do uczenia maszynowego i przetwarzania obrazów. Wiele z algorytmów dostarczanych przez biblioteke wykonywanych jest na procesorach graficznych. Api wykorzywane przez OpenCV do uruchamiania algorytmów na GPU to OpenCL. W artykule \emph{Use OpenCL in Android camera preview based CV application} \cite{OpenCV} opisano, że użycie OpenCV-T Api, które wewnątrz używa OpenCL, przy modyfikowaniu obrazu z kamery, poprawia ilość wyświetlanych klatek na sekundę. Wynik osiągany przez urządzenie Sony Xperia Z3, gdy modyfikacjaobrazu odbywała się przy pomocy kodu w C/C++ to 3-5 klatek na sekundę. Podczas używania OpenCV-T ilość klatek wzrosła do 11-13.
\subsection[Adobe]{Adobe}
Adobe to producent oprogramowania, służącego między innymi do obróbki zdjęć i materiałów wiedo. W przypadk produktów Adobe tworzonych na komputery z systemem Windows, api OpenCL jest często używane do uruchomienia przetwarzania na proesorach graficznych. W przypadku aplikacji tworzoych na system Android, wykożystywanym api jest Vulkan. Jednak według wypowiedzi Erica Berdahla \cite{IWOCL} produkty Adobe na Androidzie, wykożystują kernele w języku OpenCL C, napisane dla komupterowych wersji programu. Kernele te są kompilowane do kodu pośredniego (SPIRV) a następnie wykonywane przy użyciu api Vulkana.
\subsection[Testy mierzące wydajność i jakość urządzenia]{Testy mierzące wydajność i jakość urządzenia}
Najwięcej aplikacji, które w najwidoczniejszy sposób deklarują używanie api OpenCL, to testy które mierzą wydajność urządzenia. Programy te najczęściej mają na celu zmierzenie mocy obliczeniowej czy uzyskanie podstawowych informacji o urządzeniu takich jak ilośc jednostek wykonawczych. Pozyskiwane, przez te programy dane, mają wartość informacyjną o specyfice testowanego urządzenia.

