\section[Testy Performancowe OpenCL na Mobilkach]{Testy Performancowe OpenCL na Mobilkach}

\subsection[Pierwsza sekcja]{Pierwsza sekcja}
\subsubsection{Podsekcja}
\lipsum[6]

\begin{figure}[H]
	\includegraphics[scale=0.8]{imgs/eti.png}
	\caption{Logo WETI}
\end{figure}

\lipsum[3]

\begin{figure}[H]
	\includegraphics[scale=0.8]{imgs/pg.jpg}
	\caption{Logo PG}
\end{figure}

\lipsum[7]

\subsubsection{I kolejna podsekcja}
\lipsum[2]

\begin{table}
\caption{Przykład krótkiej tabeli}
\label{tabela_1}
\begin{tabu} to \textwidth {| p{2cm} | p{2cm} | X |}
\hline
\rowcolor{lightgray} Nagłówek 1 & Nagłówek 2 & Nagłówek 3 \\ \hline
1 & lubię plaki & \lipsum[2] \\ \hline
2 & a ja nie & \lipsum[2] \\ \hline
\end{tabu}
\end{table}

\lipsum[3]

\subsection[Druga sekcja]{Druga sekcja}
\subsubsection{Podsekcja sekcji drugiej}
\lipsum[4]

\begin{longtabu} to \textwidth {| p{2cm} | X |}
\caption{Przykład długiej tabeli} \label{tabela_2} \\[-2mm] %-2mm jest wymagane, poniewaz caption w longtabu jest uznawany jako wiersz i działa na niego ustawione ograniczenie min. 2mm odstępu przed następnym wierszem
\hline
\rowcolor{lightgray} Nagłówek 1 & Nagłówek 2 \\ \hline
\endfirsthead

\rowcolor{lightgray} Nagłówek 1 & Nagłówek 2 \\ \hline
\endhead

1 & \lipsum[5] \\ \hline
2 & \lipsum[5] \\ \hline
3 & \lipsum[5] \\ \hline
4 & \lipsum[5] \\ \hline
5 & \lipsum[5] \\ \hline
6 & \lipsum[5] \\ \hline
\end{longtabu}

\lipsum[3]

\subsubsection{Kod}
\lipsum[2]

\lstinputlisting{listings/main.cpp}

\lipsum[2]
